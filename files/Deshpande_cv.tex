%LaTeX file for resume 
% This file uses the resume document class (res.cls)

\documentclass[margin]{res}
%\usepackage{helvetica} % uses helvetica postscript font (download helvetica.sty)
%\usepackage{newcent}   % uses new century schoolbook postscript font  
\topmargin=-0.1in  % start text higher on the page
% \setlength{\textheight}{10in} % increase text height to fit resume on 1 page

%\usepackage{fancyhdr}
%\pagestyle{fancy}
%\fancyhead{}
%\fancyfoot[R]{\thepage}

\usepackage{hyperref}


\pagestyle{plain}

\def\skd{\textbf{Deshpande, S.K.}}

\begin{document}  
\name{\hspace{0.05in} Sameer K. Deshpande \\  \hspace{0.45in} November 2022}
                        
\begin{resume}                        
  
  \section{CONTACT} 
  1300 University Ave., 7225B Medical Sciences Center, Madison, WI 53706 \\
  Phone: 608-262-3609 \\ 
  Primary Email: sameer.deshpande@wisc.edu \\
  Secondary Email: sameerd@alum.mit.edu \\
  Website: \texttt{https://skdeshpande91.github.io}

  \section{RESEARCH INTERESTS}
  Bayesian hierarchical modeling. Bayesian treed regression. Model selection. Causal inference. Applications in public health and sports. 
              
\section{EMPLOYMENT} 

\textbf{University of Wisconsin--Madison}, Dept.~of Statistics \hfill Madison, WI \\
\textit{Assistant Professor} \hfill August 2021 -- present

\textbf{Massachusetts Institute of Technology}, CSAIL \hfill Cambridge, MA \\
\emph{Postdoctoral Associate} \hfill September 2018 -- August 2021 \\
Supervisor: Tamara Broderick
              
\section{EDUCATION}      
                {\bf University of Pennsylvania}, The Wharton School \hfill Philadelphia, PA \\
                Ph.D., Statistics \hfill May 2018 \\
		Thesis Title: ``Bayesian model selection and estimation without MCMC'' \\
		Thesis Supervisors: Ed George and Veronika Ro\v{c}kov\'{a}
                
                {\bf Massachusetts Institute of Technology} \hfill Cambridge, MA \\
                S.B., Mathematics \hfill June 2013 
  
  
\section{PRE-PRINTS}

\skd. (2022). ``A new BART prior for flexible modeling with categorical predictors.'' \href{https://arxiv.org/abs/2211.04459}{[arXiv:2211.04459]}.

Kokandakar, A.H., Kang, H., and \skd. (2022+). ``Sensitivity of Bayesian causal forests to modeling choices: A re-analysis of the 2022 ACIC Data Challenge.'' \href{https://arxiv.org/abs/2211.02020}{[arXiv:2211.02020]}.

Brill, R.S., \skd, Wyner, A.J. (2022+). ``A Bayesian analysis of the time through the order penalty in baseball.'' \href{https://arxiv.org/abs/2210.06724}{[arXiv:2210.06724]}.

Shen, Y. and \skd. (2022+). ``On the posterior contraction of the multivariate spike-and-slab LASSO.'' \href{https://arxiv.org/abs/2209.04389}{[arXiv:2209.04389]}.

Shen, Y., Sol\'{i}s-Lemus, C., and \skd. (2022+). ``Sparse Gaussian chain graphs with the spike-and-slab LASSO: Algorithms and asymptotics.'' \href{https://arxiv.org/abs/2207.07020}{[arXiv:2207.07020]}.

Balocchi, C., \skd, George, E.I., and Jensen, S.T. (2022+). ``Crime in Philadelphia: Bayesian clustering with particle optimization.'' \href{https://arxiv.org/abs/1912.00111}{[arXiv:1912.00111]}.

\skd, Bai, R., Balocchi, C., Starling, J.E., and Weiss, J. (2022+). ``VCBART: Bayesian trees for varying coefficients.'' \href{https://arxiv.org/abs/2003.06416}{[arXiv:2003.06416]}. \\ Code available at \url{https://github.com/skdeshpande91/VCBART}.


\section{PUBLICATIONS}

Trippe, B.L., \skd, and Broderick, T. (2022). ``Confidently comparing estimators with the c-value.'' \textit{Journal of the American Statistical Association}. (accepted). \href{https://arxiv.org/abs/2102.09705}{[arXiv:2102.09705]}. 

Lin, Y., Heng, S., Anand, S., \skd, and Small, D.S. (2022). ``Hemoglobin levels among male agricultural workers: analyses from the Demographic and Health Surveys to investigate a marker for chronic kidney disease of uncertain etiology.'' \textit{Journal of Occupational and Environmental Medicine}. (accepted). \\ \href{https://www.medrxiv.org/content/10.1101/2021.09.14.21263584v2}{[medRxiv:2021.09.14.21263584}.

Stephenson, W.T., Ghosh, S., Nguyen, T.D., Yurochkin, M, \skd,and Broderick, T. (2022). ``Measuring the robustness of Gaussian processes to kernel choice.'' \textit{AISTATS 2022}. \href{https://arxiv.org/abs/2106.06510}{[arXiv:2106.06510]}

Jin, S., Rabinowitz, A.R., Weiss, J., \skd, Gupta, N., May, R.A.B., and Small, D.S. (2021). ``Retrospective survey of youth sports participation: development and assessment of reliability using school records.'' \textit{PLOS ONE}. 16(9): e0257487. \href{https://doi.org/10.1371/journal.pone.0257487}{DOI:10.1371/journal.pone.0257487}.

Weiss, J., Rabinowitz, A.R., \skd, Hasegawa, R.B., and Small, D.S. (2021). ``Participation in collision sports and cognitive aging among Swedish Twins.'' \textit{American Journal of Epidemiology}. 190(12): 2604--2611. \href{https://doi.org/10.1093/aje/kwab177}{DOI:10.1093/aje/kwab177}.

Ghosh, S., Stephenson, W.T., Nguyen, T.D., \skd, and Broderick, T. (2020). ``Approximate cross-validation for structured models.'' \textit{NeurIPS 2020} \href{https://arxiv.org/abs/2006.12669}{[arXiv:200612669]}. 

Hasegawa, R.B, \skd, Rosenbaum, P.R., Small, D.S. (2020). ``Causal inference with two versions of treatment.'' \textit{Journal of Educational and Behavioral Statistics}. 45(4): 426 -- 445. \href{https://doi.org/10.3102/1076998620914003}{DOI: 10.3102/1076998620914003}. \href{https://arxiv.org/abs/1705.03918}{[arXiv:1705.03918]}

\skd, Hasegawa, R.B., Weiss, J., and Small, D.S. (2020). ``The association between football participation in adolescence and mental health in early adulthood.'' \textit{PLOS ONE}. 15(3): 1--14. \href{https://doi.org/10.1371/journal.pone.0229978}{DOI: 10.1371/journal.pone.0229978}.

\skd and Evans, K.E. (2020). ``Expected hypothetical completion probability.'' \textit{Journal of Quantitative Analysis in Sports}. 16(2): 85-- 94.\\ \href{https://doi.org/10.1515/jqas-2019-0050}{DOI: 10.1515/jqas-2019-0050}. \href{https://arxiv.org/abs/1910.12337}{[arXiv:1910.12337]}.

Gaulton, T.G., \skd, Small, D.S., Neuman, M.D. (2020). ``Observational study of the association between participation in high school football and self-rated health, obesity, and pain in late adulthood.'' \textit{American Journal of Epidemiology}. 186(6): 592 -- 601. \href{https://doi.org/10.1093/aje/kwz260}{DOI: 10.1093/aje/kwz260}.

\skd, Ro\v{c}kov\'{a}, V.,  George, E.I. (2019) ``Simultaneous variable and covariance selection with the multivariate spike-and-slab LASSO.'' \textit{Journal of Computational and Graphical Statistics}. 28(4): 921--931. \href{https://doi.org/10.1080/10618600.2019.1593179}{DOI:10.1080/10618600.2019.1593179}. \href{https://arxiv.org/abs/1708.08911}{[arXiv:1708.08911]}. \\ Code available at: \url{https://github.com/skdeshpande91/multivariate_SSL}

\skd and Wyner, A.J. (2017). ``A hierarchical Bayesian model of pitch framing.'' \textit{Journal of Quantitative Analysis in Sports}. 13(3): 95--112. \textbf{Editor's Choice article}.   \href{https://doi.org/10.1515/jqas-2017-0027}{DOI: 10.1515/jqas-2017-0027}. \href{https://arxiv.org/abs/1704.00823}{[arXiv:1704.00823]}.

\skd, Hasegawa, R.B., Rabinowitz, A.R., Whyte, J., Roan, C.L., Tabatabaei, A., Baiocchi, M., Karlawish, J.H., Master, C.L., and Small, D.S. (2017). ``Association of playing high school football with cognition and mental health later in life.'' \textit{JAMA Neurology}. 74(8): 909--918. \href{https://doi.org/10.1001/jamaneurol.2017.1317}{DOI:10.1001/jamaneurol.2017.1317}.

\skd and Jensen, S.T. (2016). ``Estimating an NBA player's impact on his team's chances of winning," \textit{Journal of Quantitative Analysis in Sports}. 12(2): 51--72. \textbf{Editor's Choice article}.  \href{https://doi.org/10.1515/jqas-2015-0027}{DOI:10.1515/jqas-2015-0027}.\href{https://arxiv.org/abs/1604.03186}{[arXiv:1604.03186]}

%\section{WORKING PAPERS \& PAPERS IN PROGRESS}

\section{HONORS \& AWARDS} 

Significant Contributor Award, ASA Section on Statistics in Sports (2021)

Third Prize, Ruth and William Silen, M.D. Poster Award, New England Science Symposium (2019)

Finalist, National Football League Big Data Bowl (2019)

Deming Student Scholar Award, Deming Conference on Applied Statistics. (2017)

J. Parker Bursk Memorial Award for excellence in research, Statistics Department, Wharton. (2017)

Donald S. Murray Prize for excellence in teaching, Statistics Department, Wharton (2016)

Wharton Doctoral Program Fellowship, Wharton (2013).

Travel Awards: O'Bayes (2017), BNP12 (2019), O'Bayes (2019), Bayes Comp (2020)

\section{TEACHING}

\textbf{University of Wisconsin -- Madison} 

STAT 775: {\it Introduction to Bayesian Decision and Control I} \hfill Spring 2022, Fall 2022

STAT 479: {\it Introduction to Bayesian Data Analysis} \hfill Fall 2021

\section{DEPARTMENT \hspace{0.1in} SEMINARS}

Beyond axis-aligned decision rules: Bayesian treed regression with structured categorical inputs -- Iowa State University (Statistics, September 2022), University of Chicago (Econometrics \& Statistics, December 2022), University of Pennsylvania (Biostatistics, December 2022)

The Multivariate Spike-and-Slab LASSO -- Loyola University of Chicago (Mathematics \& Statistics, October 2021)

Revisiting pitch framing with Bayesian Additive Regression Trees -- University of Virginia (Sports Analytics Lab, September 2021)

VCBART: Bayesian trees for varying coefficients -- Boston University (Biostatistics, December 2020), Wake Forest University (Mathematics, December 2020), University of Washington (Statistics, January 2021), 
Yale (Statistics \& Data Science, January 2021), Texas A\&M University (Statistics, January 2021), National University of Singapore (Statistics, January 2021), Texas State University (Mathematics, January 2021), University of Wisconsin -- Madison (February 2021). 

Estimating the health consequences of playing football using observational data -- University of St. Thomas (September 2020)

Bayesian clustering with particle optimization -- UT Austin (January 2020), LSU HSC New Orleans (January 2020)

\section{CONFERENCE \hspace{0.1in} TALKS}

Beyond axis-aligned decision rules: Bayesian treed regression with structured categorical inputs. -- ISBA 2022, CMStatistics 2022, BayesComp 2023

Revisiting pitch framing with Bayesian Additive Regression Trees -- EURO 2022, JSM 2021.

VCBART: Bayesian trees for varying coefficients -- JSM 2020, SBIES 2021, ESOBE 2021.

Approximate multiple shrinkage for clustered regression -- BNP 2019.

Bayesian spatial clustering with particle optimization -- JSM 2018. 

Expected hypothetical completion probability -- CMU Sports Analysis Conference 2019.

Estimating the health consequences of playing football using observational data: challenges, lessons learned, and new directions -- JSM 2019, Ohio State Bayesian Causal Inference Workshop 2019, NESS 2019, CMU Sports Analytics Conference 2018.

Simultaneous variable and covariance selection with the multivariate spike-and-slab LASSO -- ISBA 2018, Eco Sta 2018, BayesComp 2018, SBIES 2018, CMStatistics 2017, JSM 2017.

A hierarchical model of pitch framing -- JSM 2016, NESSIS 2015.

Estimating an NBA player's impact on his team's chances of winning -- JMM 2015, JSM 2014.

\section{SERVICE}

\textbf{Student supervision}: Yunyi Shen, Ryan Yee, Mingya Huang, Ajinkya Kokandakar (UW--Madison)

\textbf{Ph.D. dissertation committee}: Chan Park, Peng Yu, Tun Lee Ng, Yanbo Shen, Kehui Yao (UW--Madison).

\textbf{Workshop organizer}: 

Your model is wrong: robustness \& misspecification in probabilistic modeling \\
NeurIPS 2021 Workshop \hfill December 2021

Perspectives in statistical modeling and inference \\
A workshop in honor of Ed George's 70th birthday \hfill December 2021


\textbf{Journal Reviewer:} Journal of the American Statistical Association, Annals of Applied Statistics, Journal of Machine Learning Research, Journal of Computational and Graphical Statistics, Bayesian Analysis, Spatial Statistics, Journal of Multivariate Analysis, Statistics and Computation, Journal of Quantitative Analysis in Sport, Australian \& New Zealand Journal of Statistics, PLOS ONE, Harvard Data Science Review, STAT. 

\textbf{Conference Reviewer}: BNP@NeurIPS 2018, UAI 2019, AAAI 2020,  NeurIPS (2019--2022), AISTATS (2019--2022), ICML (2019, 2022).
\end{resume} 
\end{document}










%%% Local Variables:
%%% mode: latex
%%% TeX-master: t
%%% End:
